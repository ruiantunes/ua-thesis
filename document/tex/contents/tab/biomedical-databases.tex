\afterpage{
\begingroup
% \normalfont
% \color{black}
\singlespacing

% \renewcommand{\arraystretch}{1.2}
\setlength\tabcolsep{4.0pt}

% \newcommand{\z}{\hspace*{1em}}
\newcommand{\minorfootnotesize}{\fontsize{9.7pt}{11.64pt}\selectfont}

\small
% \footnotesize
% \scriptsize
% \minorscriptsize
% \tiny

% O   + E    + I*2    + E
% 0.0 + 0.38 + 0.02*2 + 0.58 = 1.0

% \begin{longtable}[c]{OE{0.38\textwidth}IIE{0.58\textwidth}N}
\begin{longtable}[c]{E{0.29\textwidth}p{0.67\textwidth}}

\caption{A list of biomedical databases, ontologies, and terminologies.}
\label{tab:biomedical-databases}\\

\toprule

Resource & Description\\
% \textbf{Resource} & \textbf{Description}\\

\midrule

\as{biogrid}\newline{\footnotesize\parencite{chatraryamontri2017a}} & Biological General Repository for Interaction Datasets (\as{biogrid}) is a database containing protein, genetic, and chemical interactions.\newline{\minorfootnotesize\url{https://thebiogrid.org}}\\

\midrule

\as{ctd}\newline{\footnotesize\parencite{davis2017a}} & Comparative Toxicogenomics Database (\as{ctd}) contains information about interactions between chemicals, genes, and diseases.\newline{\minorfootnotesize\url{https://ctdbase.org}}\\

\midrule

DrugBank\newline{\footnotesize\parencite{wishart2018a}} & DrugBank is a database containing molecular information about drugs, their mechanisms, interactions, and targets.\newline{\minorfootnotesize\url{https://www.drugbank.com}}\\

\midrule

\as{mesh}\newline{\footnotesize\parencite{lipscomb2000a}} & Medical Subject Headings (\as{mesh}) is the terminology used for indexing articles for \as{medline} literature.\newline{\minorfootnotesize\url{https://www.ncbi.nlm.nih.gov/mesh}}\\

\midrule

\as{mimic}-III\newline{\footnotesize\parencite{johnson2016a}} & Medical Information Mart for Intensive Care (\as{mimic}) is a large database comprising information, such as medications and clinical notes, about patients admitted to critical care units.\newline{\minorfootnotesize\url{https://mimic.mit.edu}}\\

\midrule

\as{obo} Foundry\newline{\footnotesize\parencite{smith2007a}} & The Open Biological and Biomedical Ontologies (\as{obo}) Foundry is an initiative for maintaining a family of interoperable ontologies in the biomedical domain.\newline{\minorfootnotesize\url{https://obofoundry.org}}\\

\midrule

\as{pubmed}\newline{\footnotesize\parencite{sayers2021a}} & \as{pubmed} (Public \as{medline}) is a database comprising scientific abstracts and citations published in life science journals.\newline{\minorfootnotesize\url{https://pubmed.ncbi.nlm.nih.gov}}\\

\midrule

\as{snomed}\newline{\footnotesize\parencite{cornet2008a}} & Systematized Nomenclature of Medicine (\as{snomed}) is a global standard for clinical health terminology.\newline{\minorfootnotesize\url{https://www.snomed.org}}\\

\midrule

\as{umls}\newline\footnotesize{\parencite{mccray1989a}} & Unified Medical Language System (\as{umls}) integrates biomedical terminology, coding standards, and other resources such as a semantic network for improving interoperability between biomedical information systems.\newline{\minorfootnotesize\url{https://www.nlm.nih.gov/research/umls/index.html}}\\

\midrule

\as{uniprot}\newline\footnotesize{\parencite{theuniprot2017a}} & Universal Protein Resource (\as{uniprot}) is a repository containing information about protein sequences and associated information.\newline{\minorfootnotesize\url{https://www.uniprot.org}}\\

\bottomrule

\end{longtable}
\endgroup
}
