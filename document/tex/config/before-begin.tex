% Title and author. This is not mandatory to change, because this is
% only used by the "\maketitle" command which it is not used in this
% template.
\title{The title}
\author{The author}

% Table of contents depth. This is optional (comment to use default).
%   Depth of the table of contents (TOC):
%     1 ... chapter and sections;
%     2 ... chapters, sections, and subsections;
%     3 ... chapters, sections, subsections, and subsubsections.
% \setcounter{tocdepth}{3}
% \setcounter{secnumdepth}{3}

% List of abbreviations.
% \newabbreviation
{3d}
{3D}
{three-dimensional}

\newabbreviation
{ace}
{ACE}
{Automatic Content Extraction}

\newabbreviation
{acl}
{ACL}
{Association for Computational Linguistics}

\newabbreviation
{acm}
{ACM}
{Association for Computing Machinery}

\newabbreviation
{ade}
{ADE}
{adverse drug event / adverse drug effect}

\newabbreviation
{adr}
{ADR}
{adverse drug reaction}

\newabbreviation
{ai}
{AI}
{artificial intelligence}

\newabbreviation
{ann}
{ANN}
{artificial neural network}

\newabbreviation
{api}
{API}
{application programming interface}

\newabbreviation
{auc}
{AUC}
{area under the curve}

\newabbreviation
{bert}
{BERT}
{Bidirectional Encoder Representations from Transformers}

\newabbreviation
{bie}
{BIE}
{biomedical \glsxtrshort{ie}}

\newabbreviation
{bilou}
{BILOU}
{beginning, inside, last, outside, unit-length}

\newabbreviation
{bilstm}
{BiLSTM}
{bidirectional \glsxtrshort{lstm}}

\newabbreviation
{bio}
{BIO}
{beginning, inside, outside}

\newabbreviation
{biocreative}
{BioCreative}
{Critical Assessment of Information Extraction systems in Biology}

\newabbreviation
{bioes}
{BIOES}
{beginning, inside, outside, end, singleton}

\newabbreviation
{biogrid}
{BioGRID}
{Biological General Repository for Interaction Datasets}

\newabbreviation
{bioie}
{BioIE}
{biomedical \glsxtrshort{ie}}

\newabbreviation
{bionlp}
{BioNLP}
{biomedical \glsxtrshort{nlp}}

\newabbreviation
{bllip}
{BLLIP}
{Brown Laboratory for Linguistic Information Processing}

\newabbreviation
{bmc}
{BMC}
{BioMed Central}

\newabbreviation
{bmewo}
{BMEWO}
{beginning, middle, end, whole, outside}

\newabbreviation
{bow}
{BoW}
{bag-of-words}

\newabbreviation
{brat}
{BRAT}
{brat rapid annotation tool}

\newabbreviation
{cbow}
{CBoW}
{continuous \glsxtrshort{bow}}

\newabbreviation
{cdr}
{CDR}
{chemical--disease relation}

\newabbreviation
{chebi}
{ChEBI}
{Chemical Entities of Biological Interest}

\newabbreviation
{cl}
{CL}
{Cell Ontology}

\newabbreviation
{cli}
{CLI}
{command-line interface}

\newabbreviation
{clpsych}
{CLPsych}
{Computational Linguistics and Clinical Psychology}

\newabbreviation
{cnn}
{CNN}
{convolutional neural network}

\newabbreviation
{conll}
{CoNLL}
{Computational Natural Language Learning}

\newabbreviation
{covid19}
{COVID-19}
{coronavirus disease 2019}

\newabbreviation
{cpi}
{CPI}
{chemical--protein interaction}

\newabbreviation
{cpr}
{CPR}
{chemical--protein relation}

\newabbreviation
{cpu}
{CPU}
{central processing unit}

\newabbreviation
{craft}
{CRAFT}
{Colorado Richly Annotated Full-Text}

\newabbreviation
{crf}
{CRF}
{conditional random field}

\newabbreviation
{css}
{CSS}
{Cascading Style Sheets}

\newabbreviation
{csv}
{CSV}
{comma-separated values}

\newabbreviation
{ctakes}
{cTAKES}
{clinical Text Analysis and Knowledge Extraction System}

\newabbreviation
{ctd}
{CTD}
{Comparative Toxicogenomics Database}

\newabbreviation
{cui}
{CUI}
{Concept Unique Identifier}

\newabbreviation
{ddi}
{DDI}
{drug--drug interaction}

\newabbreviation
{dinto}
{DINTO}
{Drug--Drug Interactions Ontology}

\newabbreviation
{dna}
{DNA}
{deoxyribonucleic acid}

\newabbreviation
{dm}
{DM}
{data mining}

\newabbreviation
{do}
{DO}
{Disease Ontology}

\newabbreviation
{dt}
{DT}
{decision tree}

\newabbreviation
{ebi}
{EBI}
{European Bioinformatics Institute}

\newabbreviation
{efmi}
{EFMI}
{European Federation for Medical Informatics}

\newabbreviation
{ehr}
{EHR}
{electronic health record}

\newabbreviation
{elmo}
{ELMo}
{Embeddings from Language Models}

\newabbreviation
{email}
{e-mail}
{electronic mail}

\newabbreviation
{embl}
{EMBL}
{European Molecular Biology Laboratory}

\newabbreviation
{embs}
{EMBS}
{Engineering in Medicine and Biology Society}

\newabbreviation
{emerge}
{eMERGE}
{Electronic Medical Records and Genomics}

\newabbreviation
{erde}
{ERDE}
{early risk detection error}

\newabbreviation
{erisk}
{eRisk}
{early risk prediction on the Internet}

\newabbreviation
{expasy}
{ExPASy}
{Expert Protein Analysis System}

\newabbreviation
{glove}
{GloVe}
{Global Vectors}

\newabbreviation
{go}
{GO}
{Gene Ontology}

\newabbreviation
{gpro}
{GPRO}
{gene and protein related object}

\newabbreviation
{grec}
{GREC}
{Gene Regulation Event Corpus}

\newabbreviation
{gru}
{GRU}
{gated recurrent unit}

\newabbreviation
{hgmd}
{HGMD}
{Human Gene Mutation Database}

\newabbreviation
{hgnc}
{HGNC}
{\glsxtrshort{hugo} Gene Nomenclature Committee}

\newabbreviation
{hmdb}
{HMDB}
{Human Metabolome Database}

\newabbreviation
{hmm}
{HMM}
{hidden Markov model}

\newabbreviation
{html}
{HTML}
{Hypertext Markup Language}

\newabbreviation
{hugo}
{HUGO}
{Human Genome Organisation}

\newabbreviation
{i2b2}
{i2b2}
{Informatics for Integrating Biology and the Bedside}

\newabbreviation
{icd}
{ICD}
{International Classification of Diseases}

\newabbreviation
{idf}
{IDF}
{inverse document frequency}

\newabbreviation
{ie}
{IE}
{information extraction}

\newabbreviation
{ieee}
{IEEE}
{Institute of Electrical and Electronics Engineers}

\newabbreviation
{imia}
{IMIA}
{International Medical Informatics Association}

\newabbreviation
{ir}
{IR}
{information retrieval}

\newabbreviation
{it}
{IT}
{information technology}

\newabbreviation
{json}
{JSON}
{JavaScript Object Notation}

\newabbreviation
{kb}
{KB}
{knowledge base}

\newabbreviation
{kbp}
{KBP}
{knowledge base population}

\newabbreviation
{kdd}
{KDD}
{Knowledge Discovery and Data Mining}

\newabbreviation
{kegg}
{KEGG}
{Kyoto Encyclopedia of Genes and Genomes}

\newabbreviation
{knn}
{k-NN}
{k-nearest neighbors}

\newabbreviation
{lamda}
{LaMDA}
{language models for dialog applications}

\newabbreviation
{ldc}
{LDC}
{Linguistic Data Consortium}

\newabbreviation
{lll}
{LLL}
{Learning Language in Logic}

\newabbreviation
{lod}
{LOD}
{linked open data}

\newabbreviation
{loinc}
{LOINC}
{Logical Observation Identifiers Names and Codes}

\newabbreviation
{lr}
{LR}
{logistic regression}

\newabbreviation
{lstm}
{LSTM}
{long short-term memory}

\newabbreviation
{mbr}
{MBR}
{\glsxtrshort{medline}/\glsxtrshort{pubmed} Baseline Repository}

\newabbreviation
{mcn}
{MCN}
{medical concept normalization}

\newabbreviation
{meddra}
{MedDRA}
{Medical Dictionary for Regulatory Activities}

\newabbreviation
{medlars}
{MEDLARS}
{Medical Literature Analysis and Retrieval System}

\newabbreviation
{medline}
{MEDLINE}
{\glsxtrshort{medlars} Online}

\newabbreviation
{mesh}
{MeSH}
{Medical Subject Headings}

\newabbreviation
{mgi}
{MGI}
{Mouse Genome Informatics}

\newabbreviation
{mimic}
{MIMIC}
{Medical Information Mart for Intensive Care}

\newabbreviation
{ml}
{ML}
{machine learning}

\newabbreviation
{mlp}
{MLP}
{multi-layer perceptron}

\newabbreviation
{mrd}
{MRD}
{machine-readable dictionary}

\newabbreviation
{msc}
{MSc}
{Master of Science}

\newabbreviation
{muc}
{MUC}
{Message Understanding Conference}

\newabbreviation
{n2c2}
{n2c2}
{National \glsxtrshort{nlp} Clinical Challenges}

\newabbreviation
{nb}
{NB}
{naive Bayes}

\newabbreviation
{ncbi}
{NCBI}
{National Center for Biotechnology Information}

\newabbreviation
{ner}
{NER}
{named entity recognition}

\newabbreviation
{nih}
{NIH}
{National Institutes of Health}

\newabbreviation
{nist}
{NIST}
{National Institute of Standards and Technology}

\newabbreviation
{nlm}
{NLM}
{National Library of Medicine}

\newabbreviation
{nlp}
{NLP}
{natural language processing}

\newabbreviation
{nltk}
{NLTK}
{Natural Language Toolkit}

\newabbreviation
{np}
{NP}
{noun phrase}

\newabbreviation
{npmi}
{NPMI}
{normalized \glsxtrshort{pmi}}

\newabbreviation
{obo}
{OBO}
{Open Biological and Biomedical Ontologies}

\newabbreviation
{oie}
{OIE}
{open \glsxtrshort{ie}}

\newabbreviation
{omim}
{OMIM}
{Online Mendelian Inheritance in Man}

\newabbreviation
{ohnlp}
{OHNLP}
{Open Health \glsxtrshort{nlp}}

\newabbreviation
{owl}
{OWL}
{Web Ontology Language}

\newabbreviation
{pdb}
{PDB}
{Protein Data Bank}

\newabbreviation
{pharmgkb}
{PharmGKB}
{Pharmacogenomics Knowledge Base}

\newabbreviation
{phd}
{PhD}
{Doctor of Philosophy}

\newabbreviation
{phi}
{PHI}
{protected health information}

\newabbreviation
{pmc}
{PMC}
{\glsxtrshort{pubmed} Central}

\newabbreviation
{pmi}
{PMI}
{pointwise mutual information}

\newabbreviation
{pmid}
{PMID}
{\glsxtrshort{pubmed} Unique Identifier}

\newabbreviation
{po}
{PO}
{Plant Ontology}

\newabbreviation
{pos}
{PoS}
{part-of-speech}

\newabbreviation
[shortplural=pp]
{pp}
{pp}
{percentage point}

\newabbreviation
{ppi}
{PPI}
{protein--protein interaction}

\newabbreviation
{pro}
{PRO}
{Protein Ontology}

\newabbreviation
{pubmed}
{PubMed}
{Public \glsxtrshort{medline}}

\newabbreviation
{qa}
{QA}
{question answering}

\newabbreviation
{ram}
{RAM}
{random access memory}

\newabbreviation
{re}
{RE}
{relation extraction}

\newabbreviation
[shortplural=regexes]
{regex}
{regex}
{regular expression}

\newabbreviation
{relu}
{ReLU}
{rectified linear unit}

\newabbreviation
{rdf}
{RDF}
{Resource Description Framework}

\newabbreviation
{rna}
{RNA}
{ribonucleic acid}

\newabbreviation
{rnn}
{RNN}
{recurrent neural network}

\newabbreviation
{rouge}
{ROUGE}
{Recall-Oriented Understudy for Gisting Evaluation}

\newabbreviation
{rsdd}
{RSDD}
{Reddit Self-reported Depression Diagnosis}

\newabbreviation
{sdp}
{SDP}
{shortest dependency path}

\newabbreviation
{semeval}
{SemEval}
{Semantic Evaluation}

\newabbreviation
{snomed}
{SNOMED}
{Systematized Nomenclature of Medicine}

\newabbreviation
{sgd1}
{SGD}
{stochastic gradient descent}

\newabbreviation
{sgd2}
{SGD}
{\textit{Saccharomyces} Genome Database}

\newabbreviation
{so}
{SO}
{Sequence Ontology}

\newabbreviation
{sparql}
{SPARQL}
{\glsxtrshort{sparql} Protocol and \glsxtrshort{rdf} Query Language}

\newabbreviation
{spl}
{SPL}
{Structured Product Labeling}

\newabbreviation
{sql}
{SQL}
{Structured Query Language}

\newabbreviation
{srl}
{SRL}
{semantic role labeling}

\newabbreviation
{srtd}
{SR\&TD}
{Scientific Research and Technological Development}

\newabbreviation
{sts}
{STS}
{semantic textual similarity}

\newabbreviation
{svg}
{SVG}
{Scalable Vector Graphics}

\newabbreviation
{svm}
{SVM}
{support vector machine}

\newabbreviation
{sw}
{SW}
{Semantic Web}

\newabbreviation
{tac}
{TAC}
{Text Analysis Conference}

\newabbreviation
{tdm}
{TDM}
{text data mining}

\newabbreviation
{tees}
{TEES}
{Turku Event Extraction System}

\newabbreviation
{tf}
{TF}
{term frequency}

\newabbreviation
{tm}
{TM}
{text mining}

\newabbreviation
{tpu}
{TPU}
{tensor processing unit}

\newabbreviation
{trec}
{TREC}
{Text Retrieval Conference}

\newabbreviation
{tsv}
{TSV}
{tab-separated values}

\newabbreviation
{toxnet}
{TOXNET}
{Toxicology Data Network}

\newabbreviation
{uie}
{UIE}
{unsupervised \glsxtrshort{ie}}

\newabbreviation
{umls}
{UMLS}
{Unified Medical Language System}

\newabbreviation
{uniprot}
{UniProt}
{Universal Protein Resource}

\newabbreviation
{uri}
{URI}
{Uniform Resource Identifier}

\newabbreviation
{url}
{URL}
{Uniform Resource Locator}

\newabbreviation
{va}
{VA}
{Veterans Affairs}

\newabbreviation
{vp}
{VP}
{verb phrase}

\newabbreviation
{w3c}
{W3C}
{\glsxtrshort{www} Consortium}

\newabbreviation
{we}
{WE}
{word embedding}

\newabbreviation
{wlpc}
{WLPC}
{Wet Lab Protocol Corpus}

\newabbreviation
{who}
{WHO}
{World Health Organization}

\newabbreviation
{wsi}
{WSI}
{word sense induction}

\newabbreviation
{wsd}
{WSD}
{word sense disambiguation}

\newabbreviation
{www}
{WWW}
{World Wide Web}

\newabbreviation
{xml}
{XML}
{Extensible Markup Language}

% \newabbreviation
{3d}
{3D}
{Three-dimensional}

\newabbreviation
{ace}
{ACE}
{Automatic Content Extraction}

\newabbreviation
{acl}
{ACL}
{Association for Computational Linguistics}

\newabbreviation
{acm}
{ACM}
{Association for Computing Machinery}

\newabbreviation
{ade}
{ADE}
{Adverse drug event / Adverse drug effect}

\newabbreviation
{adr}
{ADR}
{Adverse drug reaction}

\newabbreviation
{ai}
{AI}
{Artificial intelligence}

\newabbreviation
{ann}
{ANN}
{Artificial neural network}

\newabbreviation
{api}
{API}
{Application programming interface}

\newabbreviation
{auc}
{AUC}
{Area under the curve}

\newabbreviation
{bert}
{BERT}
{Bidirectional Encoder Representations from Transformers}

\newabbreviation
{bie}
{BIE}
{Biomedical \glsxtrshort{ie}}

\newabbreviation
{bilou}
{BILOU}
{Beginning, inside, last, outside, unit-length}

\newabbreviation
{bilstm}
{BiLSTM}
{Bidirectional \glsxtrshort{lstm}}

\newabbreviation
{bio}
{BIO}
{Beginning, inside, outside}

\newabbreviation
{biocreative}
{BioCreative}
{Critical Assessment of Information Extraction systems in Biology}

\newabbreviation
{bioes}
{BIOES}
{Beginning, inside, outside, end, singleton}

\newabbreviation
{biogrid}
{BioGRID}
{Biological General Repository for Interaction Datasets}

\newabbreviation
{bioie}
{BioIE}
{Biomedical \glsxtrshort{ie}}

\newabbreviation
{bionlp}
{BioNLP}
{Biomedical \glsxtrshort{nlp}}

\newabbreviation
{bllip}
{BLLIP}
{Brown Laboratory for Linguistic Information Processing}

\newabbreviation
{bmc}
{BMC}
{BioMed Central}

\newabbreviation
{bmewo}
{BMEWO}
{Beginning, middle, end, whole, outside}

\newabbreviation
{bow}
{BoW}
{Bag-of-words}

\newabbreviation
{brat}
{BRAT}
{Brat rapid annotation tool}

\newabbreviation
{cbow}
{CBoW}
{Continuous \glsxtrshort{bow}}

\newabbreviation
{cdr}
{CDR}
{Chemical--disease relation}

\newabbreviation
{chebi}
{ChEBI}
{Chemical Entities of Biological Interest}

\newabbreviation
{cl}
{CL}
{Cell Ontology}

\newabbreviation
{cli}
{CLI}
{Command-line interface}

\newabbreviation
{clpsych}
{CLPsych}
{Computational Linguistics and Clinical Psychology}

\newabbreviation
{cnn}
{CNN}
{Convolutional neural network}

\newabbreviation
{conll}
{CoNLL}
{Computational Natural Language Learning}

\newabbreviation
{covid19}
{COVID-19}
{Coronavirus disease 2019}

\newabbreviation
{cpi}
{CPI}
{Chemical--protein interaction}

\newabbreviation
{cpr}
{CPR}
{Chemical--protein relation}

\newabbreviation
{cpu}
{CPU}
{Central processing unit}

\newabbreviation
{craft}
{CRAFT}
{Colorado Richly Annotated Full-Text}

\newabbreviation
{crf}
{CRF}
{Conditional random field}

\newabbreviation
{css}
{CSS}
{Cascading Style Sheets}

\newabbreviation
{csv}
{CSV}
{Comma-separated values}

\newabbreviation
{ctakes}
{cTAKES}
{Clinical Text Analysis and Knowledge Extraction System}

\newabbreviation
{ctd}
{CTD}
{Comparative Toxicogenomics Database}

\newabbreviation
{cui}
{CUI}
{Concept Unique Identifier}

\newabbreviation
{ddi}
{DDI}
{Drug--drug interaction}

\newabbreviation
{dinto}
{DINTO}
{Drug--Drug Interactions Ontology}

\newabbreviation
{dna}
{DNA}
{Deoxyribonucleic acid}

\newabbreviation
{dm}
{DM}
{Data mining}

\newabbreviation
{do}
{DO}
{Disease Ontology}

\newabbreviation
{dt}
{DT}
{Decision tree}

\newabbreviation
{ebi}
{EBI}
{European Bioinformatics Institute}

\newabbreviation
{efmi}
{EFMI}
{European Federation for Medical Informatics}

\newabbreviation
{ehr}
{EHR}
{Electronic health record}

\newabbreviation
{elmo}
{ELMo}
{Embeddings from Language Models}

\newabbreviation
{email}
{e-mail}
{Electronic mail}

\newabbreviation
{embl}
{EMBL}
{European Molecular Biology Laboratory}

\newabbreviation
{embs}
{EMBS}
{Engineering in Medicine and Biology Society}

\newabbreviation
{emerge}
{eMERGE}
{Electronic Medical Records and Genomics}

\newabbreviation
{erde}
{ERDE}
{Early risk detection error}

\newabbreviation
{erisk}
{eRisk}
{Early risk prediction on the Internet}

\newabbreviation
{expasy}
{ExPASy}
{Expert Protein Analysis System}

\newabbreviation
{glove}
{GloVe}
{Global Vectors}

\newabbreviation
{go}
{GO}
{Gene Ontology}

\newabbreviation
{gpro}
{GPRO}
{Gene and protein related object}

\newabbreviation
{grec}
{GREC}
{Gene Regulation Event Corpus}

\newabbreviation
{gru}
{GRU}
{Gated recurrent unit}

\newabbreviation
{hgmd}
{HGMD}
{Human Gene Mutation Database}

\newabbreviation
{hgnc}
{HGNC}
{\glsxtrshort{hugo} Gene Nomenclature Committee}

\newabbreviation
{hmdb}
{HMDB}
{Human Metabolome Database}

\newabbreviation
{hmm}
{HMM}
{Hidden Markov model}

\newabbreviation
{html}
{HTML}
{Hypertext Markup Language}

\newabbreviation
{hugo}
{HUGO}
{Human Genome Organisation}

\newabbreviation
{i2b2}
{i2b2}
{Informatics for Integrating Biology and the Bedside}

\newabbreviation
{icd}
{ICD}
{International Classification of Diseases}

\newabbreviation
{idf}
{IDF}
{Inverse document frequency}

\newabbreviation
{ie}
{IE}
{Information extraction}

\newabbreviation
{ieee}
{IEEE}
{Institute of Electrical and Electronics Engineers}

\newabbreviation
{imia}
{IMIA}
{International Medical Informatics Association}

\newabbreviation
{ir}
{IR}
{Information retrieval}

\newabbreviation
{it}
{IT}
{Information technology}

\newabbreviation
{json}
{JSON}
{JavaScript Object Notation}

\newabbreviation
{kb}
{KB}
{Knowledge base}

\newabbreviation
{kbp}
{KBP}
{Knowledge base population}

\newabbreviation
{kdd}
{KDD}
{Knowledge Discovery and Data Mining}

\newabbreviation
{kegg}
{KEGG}
{Kyoto Encyclopedia of Genes and Genomes}

\newabbreviation
{knn}
{k-NN}
{K-nearest neighbors}

\newabbreviation
{lamda}
{LaMDA}
{Language models for dialog applications}

\newabbreviation
{ldc}
{LDC}
{Linguistic Data Consortium}

\newabbreviation
{lll}
{LLL}
{Learning Language in Logic}

\newabbreviation
{lod}
{LOD}
{Linked open data}

\newabbreviation
{loinc}
{LOINC}
{Logical Observation Identifiers Names and Codes}

\newabbreviation
{lr}
{LR}
{Logistic regression}

\newabbreviation
{lstm}
{LSTM}
{Long short-term memory}

\newabbreviation
{mbr}
{MBR}
{\glsxtrshort{medline}/\glsxtrshort{pubmed} Baseline Repository}

\newabbreviation
{mcn}
{MCN}
{Medical concept normalization}

\newabbreviation
{meddra}
{MedDRA}
{Medical Dictionary for Regulatory Activities}

\newabbreviation
{medlars}
{MEDLARS}
{Medical Literature Analysis and Retrieval System}

\newabbreviation
{medline}
{MEDLINE}
{\glsxtrshort{medlars} Online}

\newabbreviation
{mesh}
{MeSH}
{Medical Subject Headings}

\newabbreviation
{mgi}
{MGI}
{Mouse Genome Informatics}

\newabbreviation
{mimic}
{MIMIC}
{Medical Information Mart for Intensive Care}

\newabbreviation
{ml}
{ML}
{Machine learning}

\newabbreviation
{mlp}
{MLP}
{Multi-layer perceptron}

\newabbreviation
{mrd}
{MRD}
{Machine-readable dictionary}

\newabbreviation
{msc}
{MSc}
{Master of Science}

\newabbreviation
{muc}
{MUC}
{Message Understanding Conference}

\newabbreviation
{n2c2}
{n2c2}
{National \glsxtrshort{nlp} Clinical Challenges}

\newabbreviation
{nb}
{NB}
{Naive Bayes}

\newabbreviation
{ncbi}
{NCBI}
{National Center for Biotechnology Information}

\newabbreviation
{ner}
{NER}
{Named entity recognition}

\newabbreviation
{nih}
{NIH}
{National Institutes of Health}

\newabbreviation
{nist}
{NIST}
{National Institute of Standards and Technology}

\newabbreviation
{nlm}
{NLM}
{National Library of Medicine}

\newabbreviation
{nlp}
{NLP}
{Natural language processing}

\newabbreviation
{nltk}
{NLTK}
{Natural Language Toolkit}

\newabbreviation
{np}
{NP}
{Noun phrase}

\newabbreviation
{npmi}
{NPMI}
{Normalized \glsxtrshort{pmi}}

\newabbreviation
{obo}
{OBO}
{Open Biological and Biomedical Ontologies}

\newabbreviation
{oie}
{OIE}
{Open \glsxtrshort{ie}}

\newabbreviation
{omim}
{OMIM}
{Online Mendelian Inheritance in Man}

\newabbreviation
{ohnlp}
{OHNLP}
{Open Health \glsxtrshort{nlp}}

\newabbreviation
{owl}
{OWL}
{Web Ontology Language}

\newabbreviation
{pdb}
{PDB}
{Protein Data Bank}

\newabbreviation
{pharmgkb}
{PharmGKB}
{Pharmacogenomics Knowledge Base}

\newabbreviation
{phd}
{PhD}
{Doctor of Philosophy}

\newabbreviation
{phi}
{PHI}
{Protected health information}

\newabbreviation
{pmc}
{PMC}
{\glsxtrshort{pubmed} Central}

\newabbreviation
{pmi}
{PMI}
{Pointwise mutual information}

\newabbreviation
{pmid}
{PMID}
{\glsxtrshort{pubmed} Unique Identifier}

\newabbreviation
{po}
{PO}
{Plant Ontology}

\newabbreviation
{pos}
{PoS}
{Part-of-speech}

\newabbreviation
[shortplural=pp]
{pp}
{pp}
{Percentage point}

\newabbreviation
{ppi}
{PPI}
{Protein--protein interaction}

\newabbreviation
{pro}
{PRO}
{Protein Ontology}

\newabbreviation
{pubmed}
{PubMed}
{Public \glsxtrshort{medline}}

\newabbreviation
{qa}
{QA}
{Question answering}

\newabbreviation
{ram}
{RAM}
{Random access memory}

\newabbreviation
{re}
{RE}
{Relation extraction}

\newabbreviation
[shortplural=regexes]
{regex}
{regex}
{Regular expression}

\newabbreviation
{relu}
{ReLU}
{Rectified linear unit}

\newabbreviation
{rdf}
{RDF}
{Resource Description Framework}

\newabbreviation
{rna}
{RNA}
{Ribonucleic acid}

\newabbreviation
{rnn}
{RNN}
{Recurrent neural network}

\newabbreviation
{rouge}
{ROUGE}
{Recall-Oriented Understudy for Gisting Evaluation}

\newabbreviation
{rsdd}
{RSDD}
{Reddit Self-reported Depression Diagnosis}

\newabbreviation
{sdp}
{SDP}
{Shortest dependency path}

\newabbreviation
{semeval}
{SemEval}
{Semantic Evaluation}

\newabbreviation
{snomed}
{SNOMED}
{Systematized Nomenclature of Medicine}

\newabbreviation
{sgd1}
{SGD}
{Stochastic gradient descent}

\newabbreviation
{sgd2}
{SGD}
{\textit{Saccharomyces} Genome Database}

\newabbreviation
{so}
{SO}
{Sequence Ontology}

\newabbreviation
{sparql}
{SPARQL}
{\glsxtrshort{sparql} Protocol and \glsxtrshort{rdf} Query Language}

\newabbreviation
{spl}
{SPL}
{Structured Product Labeling}

\newabbreviation
{sql}
{SQL}
{Structured Query Language}

\newabbreviation
{srl}
{SRL}
{Semantic role labeling}

\newabbreviation
{srtd}
{SR\&TD}
{Scientific Research and Technological Development}

\newabbreviation
{sts}
{STS}
{Semantic textual similarity}

\newabbreviation
{svg}
{SVG}
{Scalable Vector Graphics}

\newabbreviation
{svm}
{SVM}
{Support vector machine}

\newabbreviation
{sw}
{SW}
{Semantic Web}

\newabbreviation
{tac}
{TAC}
{Text Analysis Conference}

\newabbreviation
{tdm}
{TDM}
{Text data mining}

\newabbreviation
{tees}
{TEES}
{Turku Event Extraction System}

\newabbreviation
{tf}
{TF}
{Term frequency}

\newabbreviation
{tm}
{TM}
{Text mining}

\newabbreviation
{tpu}
{TPU}
{Tensor processing unit}

\newabbreviation
{trec}
{TREC}
{Text Retrieval Conference}

\newabbreviation
{tsv}
{TSV}
{Tab-separated values}

\newabbreviation
{toxnet}
{TOXNET}
{Toxicology Data Network}

\newabbreviation
{uie}
{UIE}
{Unsupervised \glsxtrshort{ie}}

\newabbreviation
{umls}
{UMLS}
{Unified Medical Language System}

\newabbreviation
{uniprot}
{UniProt}
{Universal Protein Resource}

\newabbreviation
{uri}
{URI}
{Uniform Resource Identifier}

\newabbreviation
{url}
{URL}
{Uniform Resource Locator}

\newabbreviation
{va}
{VA}
{Veterans Affairs}

\newabbreviation
{vp}
{VP}
{Verb phrase}

\newabbreviation
{w3c}
{W3C}
{\glsxtrshort{www} Consortium}

\newabbreviation
{we}
{WE}
{Word embedding}

\newabbreviation
{wlpc}
{WLPC}
{Wet Lab Protocol Corpus}

\newabbreviation
{who}
{WHO}
{World Health Organization}

\newabbreviation
{wsi}
{WSI}
{Word sense induction}

\newabbreviation
{wsd}
{WSD}
{Word sense disambiguation}

\newabbreviation
{www}
{WWW}
{World Wide Web}

\newabbreviation
{xml}
{XML}
{Extensible Markup Language}

\newabbreviation
{3d}
{3D}
{Three-Dimensional}

\newabbreviation
{ace}
{ACE}
{Automatic Content Extraction}

\newabbreviation
{acl}
{ACL}
{Association for Computational Linguistics}

\newabbreviation
{acm}
{ACM}
{Association for Computing Machinery}

\newabbreviation
{ade}
{ADE}
{Adverse Drug Event / Adverse Drug Effect}

\newabbreviation
{adr}
{ADR}
{Adverse Drug Reaction}

\newabbreviation
{ai}
{AI}
{Artificial Intelligence}

\newabbreviation
{ann}
{ANN}
{Artificial Neural Network}

\newabbreviation
{api}
{API}
{Application Programming Interface}

\newabbreviation
{auc}
{AUC}
{Area Under the Curve}

\newabbreviation
{bert}
{BERT}
{Bidirectional Encoder Representations from Transformers}

\newabbreviation
{bie}
{BIE}
{Biomedical \glsxtrshort{ie}}

\newabbreviation
{bilou}
{BILOU}
{Beginning, Inside, Last, Outside, Unit-length}

\newabbreviation
{bilstm}
{BiLSTM}
{Bidirectional \glsxtrshort{lstm}}

\newabbreviation
{bio}
{BIO}
{Beginning, Inside, Outside}

\newabbreviation
{biocreative}
{BioCreative}
{Critical Assessment of Information Extraction Systems in Biology}

\newabbreviation
{bioes}
{BIOES}
{Beginning, Inside, Outside, End, Singleton}

\newabbreviation
{biogrid}
{BioGRID}
{Biological General Repository for Interaction Datasets}

\newabbreviation
{bioie}
{BioIE}
{Biomedical \glsxtrshort{ie}}

\newabbreviation
{bionlp}
{BioNLP}
{Biomedical \glsxtrshort{nlp}}

\newabbreviation
{bllip}
{BLLIP}
{Brown Laboratory for Linguistic Information Processing}

\newabbreviation
{bmc}
{BMC}
{BioMed Central}

\newabbreviation
{bmewo}
{BMEWO}
{Beginning, Middle, End, Whole, Outside}

\newabbreviation
{bow}
{BoW}
{Bag-of-Words}

\newabbreviation
{brat}
{BRAT}
{Brat Rapid Annotation Tool}

\newabbreviation
{cbow}
{CBoW}
{Continuous \glsxtrshort{bow}}

\newabbreviation
{cdr}
{CDR}
{Chemical--Disease Relation}

\newabbreviation
{chebi}
{ChEBI}
{Chemical Entities of Biological Interest}

\newabbreviation
{cl}
{CL}
{Cell Ontology}

\newabbreviation
{cli}
{CLI}
{Command-Line Interface}

\newabbreviation
{clpsych}
{CLPsych}
{Computational Linguistics and Clinical Psychology}

\newabbreviation
{cnn}
{CNN}
{Convolutional Neural Network}

\newabbreviation
{conll}
{CoNLL}
{Computational Natural Language Learning}

\newabbreviation
{covid19}
{COVID-19}
{Coronavirus Disease 2019}

\newabbreviation
{cpi}
{CPI}
{Chemical--Protein Interaction}

\newabbreviation
{cpr}
{CPR}
{Chemical--Protein Relation}

\newabbreviation
{cpu}
{CPU}
{Central Processing Unit}

\newabbreviation
{craft}
{CRAFT}
{Colorado Richly Annotated Full-Text}

\newabbreviation
{crf}
{CRF}
{Conditional Random Field}

\newabbreviation
{css}
{CSS}
{Cascading Style Sheets}

\newabbreviation
{csv}
{CSV}
{Comma-Separated Values}

\newabbreviation
{ctakes}
{cTAKES}
{Clinical Text Analysis and Knowledge Extraction System}

\newabbreviation
{ctd}
{CTD}
{Comparative Toxicogenomics Database}

\newabbreviation
{cui}
{CUI}
{Concept Unique Identifier}

\newabbreviation
{ddi}
{DDI}
{Drug--Drug Interaction}

\newabbreviation
{dinto}
{DINTO}
{Drug--Drug Interactions Ontology}

\newabbreviation
{dna}
{DNA}
{Deoxyribonucleic Acid}

\newabbreviation
{dm}
{DM}
{Data Mining}

\newabbreviation
{do}
{DO}
{Disease Ontology}

\newabbreviation
{dt}
{DT}
{Decision Tree}

\newabbreviation
{ebi}
{EBI}
{European Bioinformatics Institute}

\newabbreviation
{efmi}
{EFMI}
{European Federation for Medical Informatics}

\newabbreviation
{ehr}
{EHR}
{Electronic Health Record}

\newabbreviation
{elmo}
{ELMo}
{Embeddings from Language Models}

\newabbreviation
{email}
{e-mail}
{Electronic Mail}

\newabbreviation
{embl}
{EMBL}
{European Molecular Biology Laboratory}

\newabbreviation
{embs}
{EMBS}
{Engineering in Medicine and Biology Society}

\newabbreviation
{emerge}
{eMERGE}
{Electronic Medical Records and Genomics}

\newabbreviation
{erde}
{ERDE}
{Early Risk Detection Error}

\newabbreviation
{erisk}
{eRisk}
{Early Risk Prediction on the Internet}

\newabbreviation
{expasy}
{ExPASy}
{Expert Protein Analysis System}

\newabbreviation
{glove}
{GloVe}
{Global Vectors}

\newabbreviation
{go}
{GO}
{Gene Ontology}

\newabbreviation
{gpro}
{GPRO}
{Gene and Protein Related Object}

\newabbreviation
{grec}
{GREC}
{Gene Regulation Event Corpus}

\newabbreviation
{gru}
{GRU}
{Gated Recurrent Unit}

\newabbreviation
{hgmd}
{HGMD}
{Human Gene Mutation Database}

\newabbreviation
{hgnc}
{HGNC}
{\glsxtrshort{hugo} Gene Nomenclature Committee}

\newabbreviation
{hmdb}
{HMDB}
{Human Metabolome Database}

\newabbreviation
{hmm}
{HMM}
{Hidden Markov Model}

\newabbreviation
{html}
{HTML}
{Hypertext Markup Language}

\newabbreviation
{hugo}
{HUGO}
{Human Genome Organisation}

\newabbreviation
{i2b2}
{i2b2}
{Informatics for Integrating Biology and the Bedside}

\newabbreviation
{icd}
{ICD}
{International Classification of Diseases}

\newabbreviation
{idf}
{IDF}
{Inverse Document Frequency}

\newabbreviation
{ie}
{IE}
{Information Extraction}

\newabbreviation
{ieee}
{IEEE}
{Institute of Electrical and Electronics Engineers}

\newabbreviation
{imia}
{IMIA}
{International Medical Informatics Association}

\newabbreviation
{ir}
{IR}
{Information Retrieval}

\newabbreviation
{it}
{IT}
{Information Technology}

\newabbreviation
{json}
{JSON}
{JavaScript Object Notation}

\newabbreviation
{kb}
{KB}
{Knowledge Base}

\newabbreviation
{kbp}
{KBP}
{Knowledge Base Population}

\newabbreviation
{kdd}
{KDD}
{Knowledge Discovery and Data Mining}

\newabbreviation
{kegg}
{KEGG}
{Kyoto Encyclopedia of Genes and Genomes}

\newabbreviation
{knn}
{k-NN}
{K-Nearest Neighbors}

\newabbreviation
{lamda}
{LaMDA}
{Language Models for Dialog Applications}

\newabbreviation
{ldc}
{LDC}
{Linguistic Data Consortium}

\newabbreviation
{lll}
{LLL}
{Learning Language in Logic}

\newabbreviation
{lod}
{LOD}
{Linked Open Data}

\newabbreviation
{loinc}
{LOINC}
{Logical Observation Identifiers Names and Codes}

\newabbreviation
{lr}
{LR}
{Logistic Regression}

\newabbreviation
{lstm}
{LSTM}
{Long Short-Term Memory}

\newabbreviation
{mbr}
{MBR}
{\glsxtrshort{medline}/\glsxtrshort{pubmed} Baseline Repository}

\newabbreviation
{mcn}
{MCN}
{Medical Concept Normalization}

\newabbreviation
{meddra}
{MedDRA}
{Medical Dictionary for Regulatory Activities}

\newabbreviation
{medlars}
{MEDLARS}
{Medical Literature Analysis and Retrieval System}

\newabbreviation
{medline}
{MEDLINE}
{\glsxtrshort{medlars} Online}

\newabbreviation
{mesh}
{MeSH}
{Medical Subject Headings}

\newabbreviation
{mgi}
{MGI}
{Mouse Genome Informatics}

\newabbreviation
{mimic}
{MIMIC}
{Medical Information Mart for Intensive Care}

\newabbreviation
{ml}
{ML}
{Machine Learning}

\newabbreviation
{mlp}
{MLP}
{Multi-Layer Perceptron}

\newabbreviation
{mrd}
{MRD}
{Machine-Readable Dictionary}

\newabbreviation
{msc}
{MSc}
{Master of Science}

\newabbreviation
{muc}
{MUC}
{Message Understanding Conference}

\newabbreviation
{n2c2}
{n2c2}
{National \glsxtrshort{nlp} Clinical Challenges}

\newabbreviation
{nb}
{NB}
{Naive Bayes}

\newabbreviation
{ncbi}
{NCBI}
{National Center for Biotechnology Information}

\newabbreviation
{ner}
{NER}
{Named Entity Recognition}

\newabbreviation
{nih}
{NIH}
{National Institutes of Health}

\newabbreviation
{nist}
{NIST}
{National Institute of Standards and Technology}

\newabbreviation
{nlm}
{NLM}
{National Library of Medicine}

\newabbreviation
{nlp}
{NLP}
{Natural Language Processing}

\newabbreviation
{nltk}
{NLTK}
{Natural Language Toolkit}

\newabbreviation
{np}
{NP}
{Noun Phrase}

\newabbreviation
{npmi}
{NPMI}
{Normalized \glsxtrshort{pmi}}

\newabbreviation
{obo}
{OBO}
{Open Biological and Biomedical Ontologies}

\newabbreviation
{oie}
{OIE}
{Open \glsxtrshort{ie}}

\newabbreviation
{omim}
{OMIM}
{Online Mendelian Inheritance in Man}

\newabbreviation
{ohnlp}
{OHNLP}
{Open Health \glsxtrshort{nlp}}

\newabbreviation
{owl}
{OWL}
{Web Ontology Language}

\newabbreviation
{pdb}
{PDB}
{Protein Data Bank}

\newabbreviation
{pharmgkb}
{PharmGKB}
{Pharmacogenomics Knowledge Base}

\newabbreviation
{phd}
{PhD}
{Doctor of Philosophy}

\newabbreviation
{phi}
{PHI}
{Protected Health Information}

\newabbreviation
{pmc}
{PMC}
{\glsxtrshort{pubmed} Central}

\newabbreviation
{pmi}
{PMI}
{Pointwise Mutual Information}

\newabbreviation
{pmid}
{PMID}
{\glsxtrshort{pubmed} Unique Identifier}

\newabbreviation
{po}
{PO}
{Plant Ontology}

\newabbreviation
{pos}
{PoS}
{Part-of-Speech}

\newabbreviation
[shortplural=pp]
{pp}
{pp}
{Percentage Point}

\newabbreviation
{ppi}
{PPI}
{Protein--Protein Interaction}

\newabbreviation
{pro}
{PRO}
{Protein Ontology}

\newabbreviation
{pubmed}
{PubMed}
{Public \glsxtrshort{medline}}

\newabbreviation
{qa}
{QA}
{Question Answering}

\newabbreviation
{ram}
{RAM}
{Random Access Memory}

\newabbreviation
{re}
{RE}
{Relation Extraction}

\newabbreviation
[shortplural=regexes]
{regex}
{regex}
{Regular Expression}

\newabbreviation
{relu}
{ReLU}
{Rectified Linear Unit}

\newabbreviation
{rdf}
{RDF}
{Resource Description Framework}

\newabbreviation
{rna}
{RNA}
{Ribonucleic Acid}

\newabbreviation
{rnn}
{RNN}
{Recurrent Neural Network}

\newabbreviation
{rouge}
{ROUGE}
{Recall-Oriented Understudy for Gisting Evaluation}

\newabbreviation
{rsdd}
{RSDD}
{Reddit Self-Reported Depression Diagnosis}

\newabbreviation
{sdp}
{SDP}
{Shortest Dependency Path}

\newabbreviation
{semeval}
{SemEval}
{Semantic Evaluation}

\newabbreviation
{snomed}
{SNOMED}
{Systematized Nomenclature of Medicine}

\newabbreviation
{sgd1}
{SGD}
{Stochastic Gradient Descent}

\newabbreviation
{sgd2}
{SGD}
{\textit{Saccharomyces} Genome Database}

\newabbreviation
{so}
{SO}
{Sequence Ontology}

\newabbreviation
{sparql}
{SPARQL}
{\glsxtrshort{sparql} Protocol and \glsxtrshort{rdf} Query Language}

\newabbreviation
{spl}
{SPL}
{Structured Product Labeling}

\newabbreviation
{sql}
{SQL}
{Structured Query Language}

\newabbreviation
{srl}
{SRL}
{Semantic Role Labeling}

\newabbreviation
{srtd}
{SR\&TD}
{Scientific Research and Technological Development}

\newabbreviation
{sts}
{STS}
{Semantic Textual Similarity}

\newabbreviation
{svg}
{SVG}
{Scalable Vector Graphics}

\newabbreviation
{svm}
{SVM}
{Support Vector Machine}

\newabbreviation
{sw}
{SW}
{Semantic Web}

\newabbreviation
{tac}
{TAC}
{Text Analysis Conference}

\newabbreviation
{tdm}
{TDM}
{Text Data Mining}

\newabbreviation
{tees}
{TEES}
{Turku Event Extraction System}

\newabbreviation
{tf}
{TF}
{Term Frequency}

\newabbreviation
{tm}
{TM}
{Text Mining}

\newabbreviation
{tpu}
{TPU}
{Tensor Processing Unit}

\newabbreviation
{trec}
{TREC}
{Text Retrieval Conference}

\newabbreviation
{tsv}
{TSV}
{Tab-Separated Values}

\newabbreviation
{toxnet}
{TOXNET}
{Toxicology Data Network}

\newabbreviation
{uie}
{UIE}
{Unsupervised \glsxtrshort{ie}}

\newabbreviation
{umls}
{UMLS}
{Unified Medical Language System}

\newabbreviation
{uniprot}
{UniProt}
{Universal Protein Resource}

\newabbreviation
{uri}
{URI}
{Uniform Resource Identifier}

\newabbreviation
{url}
{URL}
{Uniform Resource Locator}

\newabbreviation
{va}
{VA}
{Veterans Affairs}

\newabbreviation
{vp}
{VP}
{Verb Phrase}

\newabbreviation
{w3c}
{W3C}
{\glsxtrshort{www} Consortium}

\newabbreviation
{we}
{WE}
{Word Embedding}

\newabbreviation
{wlpc}
{WLPC}
{Wet Lab Protocol Corpus}

\newabbreviation
{who}
{WHO}
{World Health Organization}

\newabbreviation
{wsi}
{WSI}
{Word Sense Induction}

\newabbreviation
{wsd}
{WSD}
{Word Sense Disambiguation}

\newabbreviation
{www}
{WWW}
{World Wide Web}

\newabbreviation
{xml}
{XML}
{Extensible Markup Language}


% Math functions declaration.
% https://tex.stackexchange.com/questions/95171/how-can-i-define-user-defined-functions-in-math-mode
% https://tex.stackexchange.com/questions/130510/write-text-correctly-in-equations
\DeclareMathOperator{\score}{score}
\DeclareMathOperator{\NPMI}{NPMI}
\DeclareMathOperator{\CS}{CS}
\DeclareMathOperator{\functionf}{f}

% Declaring variables as math text alphabets.
% https://tex.stackexchange.com/questions/553236/how-do-i-format-words-as-variable-names
\newcommand\variablename[1]{\mathop{\mathrm{#1}}\nolimits}
\newcommand\variablenamevector[1]{\mathop{\mathbf{#1}}\nolimits}

\newcommand{\TP}{\variablename{TP}}
\newcommand{\TN}{\variablename{TN}}
\newcommand{\FP}{\variablename{FP}}
\newcommand{\FN}{\variablename{FN}}

\newcommand{\Precision}{\variablename{Precision}}
\newcommand{\Recall}{\variablename{Recall}}
\newcommand{\Accuracy}{\variablename{Accuracy}}
\newcommand{\Specificity}{\variablename{Specificity}}
\newcommand{\Fscore}{\variablename{F1\hbox{-}score}}

\newcommand{\CUI}{\variablename{CUI}}
\newcommand{\tvec}{\mathbf{t}}
\newcommand{\CUIvec}{\variablenamevector{CUI}}

% List of manually hyphenated words in case hyphenation is enabled
% (check the "hyphenat" package). Hyphenation in the middle is desired
% (better readability).
\hyphenation{
% English words.
ex-am-ple
mas-sa-chu-setts
test
% Portuguese words.
Gui-lher-me
bi-o-me-di-ci-na
cien-tí-fico
con-cei-tos
exem-plo
infor-mações
tes-te
}
